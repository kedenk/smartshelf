The whole concept of smart shelf follows a IoT principle. According to previously mentioned MQTT protocol the Arduino which is the main controller of the smart shelf is subscribed with a topic in the MQTT broker. The server itself publish topic based with on the identity of each drawer so that Arduino can receive the user command from the MQTT broker and send corresponding to each of the drawers. After receiving the information Arduino parse the JSON text and invoke command to glow any led based on the server side information. Similarly, The reading from sensor send to the back end in the same way the we receive information from backend, but this time the Web-app act as a subscriber to topics and Arduino publishes on that topics.  The communication tool and hardware used for communicating with server is Ethernet communication which solely operated by the ethernet card Arduino Ethernet Shied \cite{RefWorks:EthernetArduino}. We used Arduino Client for MQTT \cite{RefWorks:MQTTArduino} library for Arduino to MQTT Broker communication.
