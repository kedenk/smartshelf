There are several development work happened last few year in human computer interaction(HCI), home automation and embedded technology. A big set of these work is giving intelligence to rigid objects and allow human to communicate with them and vice-versa by applying noble HCI techniques. Moreover, post-WIMP devices also offer some features that can be integrate with the modern computer technology development(Ubiquitous computing). However, this post-WIMP GUI concept only applicable if there is a metaphor available in digital or analogue world. For example, searching the meaning of a word in digital dictionary(e.g:Smart phone dictionary). We want explain decent amount of successful research work that overlap at least in certain area with our Smart Shelf framework; However, there is no implementation or ground work fully overlap with our concept. A technical definition of our project is "Combining different interaction technique to innovate a device that follow the guideline of ubiquitous computing".The most related topic that are already known by design community are: QR code for presenting information, Automatic amount calculation, Controlling device.    
\subsection{QR code for presenting information}
Now a days application of QR code become very popular and common due to the smart phone technology. Now people don't need to type search. Pressing a key is enough to get information based on QR code. A very innovative application is using QR code in library management. In a case study ` `Application of QR Code Technology in providing Library and Information Services in Academic Libraries'' by  Sandeep
Kumar Pathak showed that important information can be presented by QR code and user can easily get all those information by scanning QR code. 
We are implying this idea in to completely different perspective. In our case every drawer will have individual QR code. Each code will represent individual information about items stored in the drawer. 
\subsection{Automatic amount calculation}
One major objective our implementation is representing empty or not empty drawer. As it's a very ground level work of many automation project, there are many project information available regarding weight measurement. However, in our project we are counting the objects based on the overall weight. We don't see this sort of work is not very common to automation community.Although, the most related work to that sub task is counting weight based on resistive sensor. An example of this work presented in $circuitdigest.com$:Arduino Weight Measurement using Load Cell and HX711 Module. Here they use Load Cell, but we will use resistive load sensor to calculate the weight signal.Also, their project does not include counting.
\subsection{Controlling Device}
The most important human to machine interaction task in this project is giving command to the system. There are several way to build up the this interaction system:one could be developing from scratch and another is building up over existing individual system. It's very common in Internet of Things (IoT) community to to use a smart phone for controlling a electronic system.For example there are lot of projects that use android devices for home automation. However, Our approach is similar but objective is completely different.
\\
\\
We see there are many existing work happened in granular level, but here we are bringing these granular ideas to build completely a new noble system.
