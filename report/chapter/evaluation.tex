 
To evaluate interactive Smart Shelf, we tested it in real world scenarios.
We found some strengths and weaknesses during the evaluation of the project. 
Moreover we did four activities during evaluation that are listed in table \ref{tab:user-study-results}.
In the following some strength and weaknesses are described and if the functionalities described in previous sections are implemented. 

\subsection{Strength} 
\begin{itemize}
  \item The drawer design is reliable and long lasting. 
  The system has a user interface for multiple users.
   \item In service mode, colour of LED's follows cultural analogy, i.e. Red (no item), Blue (amount of items is above a defined threshold) and Yellow (amount of items is under threshold) in each drawer.
   \item If there isn't any item in a drawer, system send notification email to operator and marks the drawer with a red LED.
   \item User can start service mode via the webapp that depicts the current state of Smart Shelf.
   \item More than one user can search items simultaneously.
   \item Using QR-Code user can get detailed information of items i.e. capacity of resistors or amount of items in drawer etc.
\end{itemize}

\subsection{Weakness}
\begin{itemize}
  \item LED's inside drawer is not clearly visible for visualization.
        \subsection{Solution}
         Use huge diffused LED's (10mm diameter). These are really bright so these can be seen in daytime. These have 465-470nm wavelength, 3.0-3.4V forward voltage and require 20mA current. Typical brightness is 1000mcd instead of using 5mm.
         Additionally, the LEDs could be placed outside of the drawer. 
         
  \item Load Cell carries a plastic plate inside the drawer to measure the weight of the items in it. 
  Because of the soft plastic the drawer are made of sometimes the plastic plate sticks or tilt with the drawer walls. 
  This leads to false measurements of the weights.
  		 \subsection{Solution}
  		 Drawers made of more solid material should be used. 
  \item  Power supply is a issue if higher amount of LEDs are switched on.
  	   \subsection{Solution}
		The origin of this problem is the limited power supply by the Arduino. 
		To tackle this problem an external power supply can be added to the LED circuits. 
\end{itemize}

\subsection{Best Scenarios} 
During the evaluation some scenarios has shown they satisfies the problems explained in previous sections. 
Searching an item in a shelf works best and was one of the main objectives of this project. 
Additionally, the service mode showed good results to give a visual feedback to users about the status of the drawers. 
  
\subsection{User Testing} 
To evaluate how the webapp of the Smart Shelf system performs with real users a small user studies was done. 
The results of the activities done in that study are listed in table \ref{tab:user-study-results}. 
All activities are done two times. 
The first time the user was able to make himself/herself familiar with the task. 
In the second round with an slightly other objective it is measured if the task could be fulfilled successfully in a given amount of time. 
\begin{description}
	\item[Activity 1] \hfill \\ 
	The use had the task to search for a specific item with the provided user interface. 
	\item [Activity 2] \hfill \\
	To get more information about a specific item the user had to scan the QR-Code on a specific drawer to get the datasheet. 
	\item [Activity 3] \hfill \\
	Objective of this activity was to obtain the amount of a specific item in the shelf. 
	\item [Activity 4] \hfill \\
	With this task the user should enable the service mode and give a feedback about empty drawers, half full and full drawers. 
\end{description}
%
\begin{table}
	\begin{tabular}{|c|c|}
		\hline 
		Activity No. & Successfully fulfilled task \\
		\hline 
		1 & 4  \\ 
		\hline 
		2 & 4  \\ 
		\hline 
		3 & 3  \\ 
		\hline 
		4 & 2  \\ 
		\hline 
	\end{tabular} 
	\caption{Overview of the user study results}~\label{tab:user-study-results}
\end{table}
%
We tested all these activities with four different participants and interviewed these afterwards to collect feedback. 
The recoded data is shown in this section. 
The age distribution of the participants were 20 to 29 years, were the people are either employees or students of the University of Saarland. 
\\
Overall for the basic functionality of the Smart Shelf we observed almost 100\% of successfully fulfilled task for all participants. 
All participants were able to fulfil the first two tasks within a short amount of time. 
For Activity 3 only one participant was not able to solve the task within the given time. 
After interviewing the participant we got the answer that it was not clear to the participant that he/she has first to search for an item to get the amount of items in the shelf. 
The participant was searching for an overview page of all drawers/items to get the information about the amount of items. 
To solve Activity 4 only two of four participants were able. 
Reason for that could be the menu to enable the service mode. 
To enable the service mode the user has to click the collapsable menu \grqq{}Management\grqq{} and follow the link \grqq{}Service Mode\grqq{}. 
One participant was not able to find this link. 
For the other participant the information about the meaning of the colours on the drawers was not clear to him/her. 
\\
\\
One limitation of the proposed solutions in the implementation section is the size of the drawers. 
Electronic components for weight measurement consumes a lot of space in the drawer. 
This results in a very limited space for items itself. 
Because of that reason we decided to apply the weight measurement in the prototype only to the bigger drawers. 
The small drawers contains just the board with the LEDs as can be seen in figure \ref{fig:example_drawer}. 
\\
\\
The evaluation of the project reveals some important aspects in which the Smart Shelf can be improved. 
The results described in this section show that the basic functionality reached good result with users which never got in touch with the new Smart Shelf. 
But other, more complex functionalities like the service mode need more work to improve the usability. 
During the interviews with the participants two minor improvements were proposed, too. 
If a user searches for an item, first all LEDs on drawers which are in the result list switched on their LED. 
Only after selecting the wanted item in the result list (see figure \ref{fig:webapp-mainpage}) all LEDs are switched of and only the LED on the correct drawer is switched on. 
Some users were confused about this and proposed to switch on only one LED after the selection. 
Furthermore, the buttons in the menu bar on top of the webpage are not that big. 
These could be bigger that they be more easy to detect and pressed, especially on mobile devices. 

A important point for improvement is the usability of the drawer itself. 
Currently all drawers wired with wires and can not easily taken off the drawer or for example to another place. 
In a next version of the drawers these should be wireless to handle them more easily and more convenient. 
This is also explained in the section \nameref{sec:future-work}.