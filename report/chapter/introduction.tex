Today the word \textit{Smart} is almost everywhere. There are \textit{Smart Homes} and \textit{Smart Fabrication}. 
But from the last few years, HCI researchers have developed new interactive interfaces, synthesize from different perspective of humans for example Psychology, Economically and Social that can be integrated with modern technologies (Ubiquitous technology). 
People prefer to use these technologies, but sometimes face problems to obtained desired results. 
For example, most of the time people don't like to use shelves, because it is hard to find the desired item. 
By using ubiquities technology, combined with technologies of the \textit{Smart Home} and \textit{Smart Fabrication} domain an interactive shelf, called \textit{Smart Shelf}, can be developed. 
This shelf can enhance the utility of regular shelves. 
This paper describes the approach of the implementation of the \textit{Smart Shelf}. 
Different approaches of the interactive design, how the \textit{Smart Shelf} is implemented and how to interact with it is discussed. 
One focus is the interaction of users and operators with the shelf. 
In the end there is an evaluation of the design decisions regarding the project and an outlook for further work. 
The evaluation contains a short user study, too. 

\subsection{Problem}
Most people when they hear about a shelf they think about their bookshelf or some shelves in the kitchen. 
Almost everyone who have a bookshelf searched at least one time in his/her life for a book in it and wished to have a guideline how to find it the fastest way. 
Imagine big shelves with a lot of small drawers. 
Every drawer is only labelled with a small name that describes what in that drawer is. 
Searching for items in these shelves can be hard and cost a lot of time. 
An additional scenario is if you apply this concept to big warehouses with hundreds of shelves and more drawers or places where you can place items. 
Finding an item in such a warehouse can be still harder. 

Shelves used in companies or research institutes bring more problems to the surface. 
Often there are shelves used for storage of electronic components
\footnote{e.g. Resistors, Capacitor, Micro Controllers or Integrated Circuits} for example. 
The drawers contained in these shelves are often a lot and small. 
Searching for a specific item/drawer in such a shelf with for example fifty drawers can be exhausting. 
But this is not only one problem. 
Is the searched drawer found, the contained items are maybe out of stock and the user wasted his/her time for search. 
This is not only exhausting, it is also wast of time if the user could directly see if the searched item is out of stock. 
\\
\\
Not only warehouses or storage rooms with shelves have those problems with the inefficiency in finding items or the premise if one item is out of stock. 
The same problems appear in retail. 
Customers which can't find their favourite product in a shop are unsatisfied. 
Maybe they go to another shop and don't come back. 
This problem can be tackled with \textit{Smart Shelves}, too. 
The shelf itself could detect if some products in it are only available in a small amount. 
In this case the shelf could order new products or at least send an information to an operator who can order supplies. 
With this strategy there will be no more empty shelves in shops and customers can find their favourite product all the time. 
This strategy is very similar to the current trend of automation called \textit{Industrie 4.0}.
\\
\\
\textit{Smart Shelf} should be a solution for these problems. 
It could observe the amount of items in itself, help people to find products and also order supplies if the amount of items is low. 
The project described by this paper tries to give a solution for the mentioned problems. 