\documentclass{sigchi}

% Use this section to set the ACM copyright statement (e.g. for
% preprints).  Consult the conference website for the camera-ready
% copyright statement.

% Copyright
\CopyrightYear{2017}
%\setcopyright{acmcopyright}
\setcopyright{acmlicensed}
%\setcopyright{rightsretained}
%\setcopyright{usgov}
%\setcopyright{usgovmixed}
%\setcopyright{cagov}
%\setcopyright{cagovmixed}
% DOI
%\doi{http://dx.doi.org/10.475/123_4}
% ISBN
%\isbn{123-4567-24-567/08/06}
%Conference
%\conferenceinfo{CHI'16,}{May 07--12, 2016, San Jose, CA, USA}
%Price
%\acmPrice{\$15.00}

% Use this command to override the default ACM copyright statement
% (e.g. for preprints).  Consult the conference website for the
% camera-ready copyright statement.

%% HOW TO OVERRIDE THE DEFAULT COPYRIGHT STRIP --
%% Please note you need to make sure the copy for your specific
%% license is used here!
% \toappear{
% Permission to make digital or hard copies of all or part of this work
% for personal or classroom use is granted without fee provided that
% copies are not made or distributed for profit or commercial advantage
% and that copies bear this notice and the full citation on the first
% page. Copyrights for components of this work owned by others than ACM
% must be honored. Abstracting with credit is permitted. To copy
% otherwise, or republish, to post on servers or to redistribute to
% lists, requires prior specific permission and/or a fee. Request
% permissions from \href{mailto:Permissions@acm.org}{Permissions@acm.org}. \\
% \emph{CHI '16},  May 07--12, 2016, San Jose, CA, USA \\
% ACM xxx-x-xxxx-xxxx-x/xx/xx\ldots \$15.00 \\
% DOI: \url{http://dx.doi.org/xx.xxxx/xxxxxxx.xxxxxxx}
% }

% Arabic page numbers for submission.  Remove this line to eliminate
% page numbers for the camera ready copy
% \pagenumbering{arabic}

% Load basic packages
\usepackage{balance}       % to better equalize the last page
\usepackage{graphics}      % for EPS, load graphicx instead 
\usepackage[T1]{fontenc}   % for umlauts and other diaeresis
\usepackage{txfonts}
\usepackage{mathptmx}
\usepackage[pdflang={en-US},pdftex]{hyperref}
\usepackage{color}
\usepackage{booktabs}
\usepackage{textcomp}

% Some optional stuff you might like/need.
\usepackage{microtype}        % Improved Tracking and Kerning
% \usepackage[all]{hypcap}    % Fixes bug in hyperref caption linking
\usepackage{ccicons}          % Cite your images correctly!
% \usepackage[utf8]{inputenc} % for a UTF8 editor only

% If you want to use todo notes, marginpars etc. during creation of
% your draft document, you have to enable the "chi_draft" option for
% the document class. To do this, change the very first line to:
% "\documentclass[chi_draft]{sigchi}". You can then place todo notes
% by using the "\todo{...}"  command. Make sure to disable the draft
% option again before submitting your final document.
\usepackage{todonotes}

% Paper metadata (use plain text, for PDF inclusion and later
% re-using, if desired).  Use \emtpyauthor when submitting for review
% so you remain anonymous.
\def\plaintitle{Smart Shelf: Project Proposal}
\def\plainauthor{Kevin Denk, Md. Abdul Kadir, Atika Akmal}
\def\emptyauthor{}
\def\plainkeywords{Smart Fabrication, Smart, Shelf, HCI, Physical Computing}
\def\plaingeneralterms{Project Proposal}

% llt: Define a global style for URLs, rather that the default one
\makeatletter
\def\url@leostyle{%
  \@ifundefined{selectfont}{
    \def\UrlFont{\sf}
  }{
    \def\UrlFont{\small\bf\ttfamily}
  }}
\makeatother
\urlstyle{leo}

% To make various LaTeX processors do the right thing with page size.
\def\pprw{8.5in}
\def\pprh{11in}
\special{papersize=\pprw,\pprh}
\setlength{\paperwidth}{\pprw}
\setlength{\paperheight}{\pprh}
\setlength{\pdfpagewidth}{\pprw}
\setlength{\pdfpageheight}{\pprh}

% Make sure hyperref comes last of your loaded packages, to give it a
% fighting chance of not being over-written, since its job is to
% redefine many LaTeX commands.
\definecolor{linkColor}{RGB}{6,125,233}
\hypersetup{%
  pdftitle={\plaintitle},
% Use \plainauthor for final version.
%  pdfauthor={\plainauthor},
  pdfauthor={\emptyauthor},
  pdfkeywords={\plainkeywords},
  pdfdisplaydoctitle=true, % For Accessibility
  bookmarksnumbered,
  pdfstartview={FitH},
  colorlinks,
  citecolor=black,
  filecolor=black,
  linkcolor=black,
  urlcolor=linkColor,
  breaklinks=true,
  hypertexnames=false
}

% create a shortcut to typeset table headings
% \newcommand\tabhead[1]{\small\textbf{#1}}

% End of preamble. Here it comes the document.
\begin{document}

\title{\plaintitle}

\numberofauthors{3}
\author{%
	\alignauthor{Md. Abdul Kadir\\
		\affaddr{for Submission}\\
		\affaddr{Saarbr{\"u}cken, Germany}\\
		\email{maktareq@gmail.com}}\\
  \alignauthor{Kevin Denk\\
    \affaddr{for Submission}\\
    \affaddr{Saarbr{\"u}cken, Germany}\\
    \email{s8kedenk@stud.uni-saarland.de}}\\
  \alignauthor{Atika Akmal\\
    \affaddr{for Submission}\\
    \affaddr{Saarbr{\"u}cken, Germany}\\
    \email{atikaakmal19@gmail.com}}\\
}

\maketitle

\begin{abstract}
  Saarbr{\"u}cken---\today. 
\end{abstract}

\category{H.5.m.}{Information Interfaces and Presentation
  (e.g. HCI)}{Miscellaneous} \category{See
  \url{http://acm.org/about/class/1998/} for the full list of ACM
  classifiers. This section is required.}{}{}

\keywords{\plainkeywords}

\section{Introduction}
Today the word smart is almost everywhere. 
There are smart homes and smart fabrications. 


\section{Problem}
Most people when they hear about a shelf they think about their bookshelf or some shelves in the kitchen. 
Almost everyone who have a bookshelf searched at least one time in his/her life for a book in it and wished to have a guideline how to find it the fastest way. 
Imagine big shelves with a lot of small drawers. 
Every drawer is only labelled with a small name which describes what is in that drawer. 
Searching for items in these shelves can be hard and cost a lot of time. 
An additional scenario is if you apply this concept to big warehouses with hundreds of shelves and more drawers or places where you can place items. 
Finding in such a warehouse an item can be still harder. 

Especially if you use shelves to store for example electronic components the next problem appears if you finally found the correct drawer with the searched component. 
The drawer is empty and no one ordered supplies. 
This is not only annoying, also the productivity of team sinks. 
Maybe the team or colleague can not finish work because they need that component that is not in stock currently. 
\\
\\
Not only warehouses or storage rooms with shelves have those problems with the inefficiency in finding items or the premise if one item is out of stock. 
The same problems appear in retail. 
Customer which can't find their favourite product in a shop are unsatisfied. 
Maybe they go to another shop and from that moment go directly to that other shop. 
This problem can also be tackled with smart shelves. 
The shelf itself could detect if some products in it are only available in a small amount. 
In this case the shelf could order new products or at least send an information to an operator who can order supplies. 
With this strategy there will be no more empty shelves in shops and customers can find their favourite product all the time. 

Smart Shelf should be a solution for these problems. 
It could observe the amount of items in itself, help people to find products and also order supplies if the amount of items is low. 
Furthermore, if you have a shelf with sensitive items or secrete documents, one can think about to restrict the access to some drawers. 
Drawers could be locked and only with scanning the appropriate QR-code on the drawer the drawer opens if the user is trustful. 


\section{Related Work}

There are several development work happened last few year in human computer interaction(HCI), home automation and embedded technology. A big set of these work is giving intelligence to rigid objects and allow human to communicate with them and vice-versa by applying noble HCI techniques. Moreover, post-WIMP devices also offer some features that can be integrate with the modern computer technology development(Ubiquitous computing). However, this post-WIMP GUI concept only applicable if there is a metaphor available in digital or analogue world. For example, searching the meaning of a word in digital dictionary(e.g:Smart phone dictionary). We want explain decent amount of successful research work that overlap at least in certain area with our Digital Shelves framework; However, there is no implementation or ground work fully overlap with our concept. A technical definition of our project is "Combining different interaction technique to innovate a device that follow the guideline of ubiquitous computing".The most related topic that already are known by HCI community are: QR code for presenting information, Automatic amount calculation, Smart Phone application for device control     
\subsection{QR code for presenting information}
Now a days application of QR code become very popular and common due to the smart phone technology. Now people don't need to type search. Pressing a key is enough to get information based on QR code. A very innovative application of QR code in library management. in a case study ` `Application of QR Code Technology in providing Library and Information Services in Academic Libraries'' by  Sandeep
Kumar Pathak showed that important information can be presented by QR code and user can easily get all those detailed information by scanning QR code. $http://events.iitgn.ac.in/2017/CLSTL/wp-content/uploads/2017/03/T7_SandeepPathak.pdf.$
\subsection{Automatic amount calculation}


\section{Approach}

\subsection{Inputs}
\subsection{Outputs}
\subsection{User Interaction}

\section{Expected Results}


\subsection{Time Plan}



\section{Conclusion}


\section{Acknowledgements}

Sample text: We thank all the volunteers, and all publications support
and staff, who wrote and provided helpful comments on previous
versions of this document. Authors 1, 2, and 3 gratefully acknowledge
the grant from NSF (\#1234--2012--ABC). \textit{This whole paragraph is
  just an example.}

% Balancing columns in a ref list is a bit of a pain because you
% either use a hack like flushend or balance, or manually insert
% a column break.  http://www.tex.ac.uk/cgi-bin/texfaq2html?label=balance
% multicols doesn't work because we're already in two-column mode,
% and flushend isn't awesome, so I choose balance.  See this
% for more info: http://cs.brown.edu/system/software/latex/doc/balance.pdf
%
% Note that in a perfect world balance wants to be in the first
% column of the last page.
%
% If balance doesn't work for you, you can remove that and
% hard-code a column break into the bbl file right before you
% submit:
%
% http://stackoverflow.com/questions/2149854/how-to-manually-equalize-columns-
% in-an-ieee-paper-if-using-bibtex
%
% Or, just remove \balance and give up on balancing the last page.
%
\balance{}

% BALANCE COLUMNS
\balance{}

% REFERENCES FORMAT
% References must be the same font size as other body text.
\bibliographystyle{SIGCHI Reference-Format}
\bibliography{sample}



\end{document}

%%% Local Variables:
%%% mode: latex
%%% TeX-master: t
%%% End:
